% !TEX root = mythesis.tex

%==============================================================================
\chapter{Conclusion}
\label{sec:conclusion}
%==============================================================================

In this thesis, we have used lattice stochastic methods to find the spectrum of the exciton. This electron-hole pair has been in recent times an interesting research topic because of the possible technological applications. The model that we used to describe these quasi-particles is the Hubbard model which we have applied on a hexagonal bipartite lattice. The honeycomb lattice is a model for graphene monolayers, and different boundary conditions lead to different carbon nanostructures e.g. graphene sheets, carbon nanotubes, and ribbons.

The methods that we have used to measure our observables culminate in the Hybrid Monte Carlo method. This is a powerful algorithm, mainly used in heavy LQCD simulations, that allows us to perform non-perturbative calculations. We have written a custom high-performance computing algorithm which aims to generate the correlation function data. After ensuring that the code produces the correct distributions, we used the JURECA cluster to generate the raw data. Once the simulations were completed, we ran a series of data reduction and management scripts. A final analysis of the processed data was then performed by fitting the one-body and two-body correlators separately. Interesting results were obtained when plotting the exciton binding energies.

First, we have extrapolated the extracted continuum binding energies to the infinite volume limit at finite temperature. We found the values at this limit to be consistently above zero (approximately $\Delta E \approx 1.5$) across all correlators and irreducible representations. Moreover, we found a small scaling while reaching this limit, meaning we can have more confidence in the results, when $\beta \to \infty$ at finite volume but with added systematic error. 

After that, we looked at the zero-temperature limit but at a finite volume. We have obtained results for two coupling strengths cases. First case is when $U$ is below the critical coupling, and we do not have a band gap. The results showed that the binding energies of $C_{phhp}$ and $C_{pppp}$ tend towards the statistical zero. Unfortunately, the uncertainties of the binding energies were too large to give a definitive answer about the existence of bound states. In the second case, we examined data with a band gap in the system. Interestingly, from the plots of $C_{phhp}$ for $U=4.0$, we observed an excitonic bound state with operators transforming within the $A1$ irreducible representation. In contrast, we did not find a bound state with antisymmetric operators ($B2$). The other correlation function $C_{pppp}$ have binding energies for both irreducible representations near or equal to zero, within the uncertainties. 

Overall, the results presented in this work showed that exciton bound states most likely exist when there is a gap and the operator is transforming within the $A1$ irreducible representation. The binding energy of this state at zero temperature and lattice size of $(6\times 6)$, depending on the scheme of extrapolation, is found to be $\Delta E^{line}_{phhp} = -\:0.938(67)$ or $\Delta E^{quad}_{phhp} = -\:0.35(34)$.

% The next steps in this project are to write more data analysis tools that will allow for better statistics and hopefully give us more insight into the energy spectrum of excitons.
% except for $C_{phhp}$ with $B2$, where $\Delta E_{phhp}$ is close to zero
% Future: the data analysis can be done with simultaneous fit